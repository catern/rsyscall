% \documentclass[sigplan]{acmart}
\documentclass{acmart}

\usepackage{hyperref}
\usepackage{float}
\usepackage{listings}
\usepackage{booktabs} % For formal tables


% Copyright
%\setcopyright{none}
%\setcopyright{acmcopyright}
%\setcopyright{acmlicensed}
\setcopyright{rightsretained}
%\setcopyright{usgov}
%\setcopyright{usgovmixed}
%\setcopyright{cagov}
%\setcopyright{cagovmixed}


% DOI
%% \acmDOI{10.475/123_4}

% ISBN
%% \acmISBN{123-4567-24-567/08/06}

%Conference
\acmConference[OSDI 2020]{SIGBOVIK}{November 2020}{Banff, Alberta, Canada}
\acmYear{2020}
\copyrightyear{2020}


%\acmBadgeL[http://ctuning.org/ae/ppopp2016.html]{ae-logo}
%\acmBadgeR[http://ctuning.org/ae/ppopp2016.html]{ae-logo}

% Document starts
\begin{document}

% Title portion. Note the short title for running heads
\title[Direct-style process creation on Linux]{Direct-style process creation on Linux}

\author{Spencer Baugh}
\affiliation{%
  \institution{Two Sigma}
  \streetaddress{100 6th Avenue}
  \city{New York City, New York}
  \country{USA}}
\email{sbaugh@twosigma.com}

\begin{abstract}
Traditional process creation interfaces,
such as fork and spawn,
are complex to use, limited in power, and difficult to abstract over.
We develop a new process creation interface for Linux
which allows a program to create a child process in an non-running state
and initialize the new process by operating on it from the outside.
This method of process creation results in more comprehensible programs, 
has better error handling,
is more efficient,
and is more amenable to abstraction.
Our implementation is immediately deployable without kernel modifications on any recent Linux kernel version.
\end{abstract}

%
% The code below should be generated by the tool at
% http://dl.acm.org/ccs.cfm
% Please copy and paste the code instead of the example below.
%
\begin{CCSXML}
<ccs2012>
   <concept>
       <concept_id>10011007.10010940.10010941.10010949.10010957</concept_id>
       <concept_desc>Software and its engineering~Process management</concept_desc>
       <concept_significance>500</concept_significance>
       </concept>
   <concept>
       <concept_id>10011007.10010940.10010941.10010949.10010950.10010951</concept_id>
       <concept_desc>Software and its engineering~Virtual memory</concept_desc>
       <concept_significance>300</concept_significance>
       </concept>
   <concept>
       <concept_id>10010147.10010169</concept_id>
       <concept_desc>Computing methodologies~Parallel computing methodologies</concept_desc>
       <concept_significance>500</concept_significance>
       </concept>
 </ccs2012>
\end{CCSXML}

\ccsdesc[500]{Software and its engineering~Process management}
\ccsdesc[300]{Software and its engineering~Virtual memory}
\ccsdesc[500]{Computing methodologies~Parallel computing methodologies}

\keywords{Process creation, Linux, fork, spawn}

\maketitle

\section{Introduction}\label{introduction}
Existing process creation interfaces can be divided into three categories:
Fork-style, spawn-style, and direct-style.

Fork-style (such as \texttt{fork}) and spawn-style (such as \texttt{posix\_spawn}) each have advantages and disadvantages.
Correct and performant usage of fork-style places significant constraints on the calling process.
Spawn-style lifts many of these constraints,
but is fundamentally limited in the kinds of processes it can create.
With both fork-style and spawn-style,
it is difficult to
branch based on results or errors encountered while initializing the new process:
fork-style requires IPC if it wishes to communicate errors back to the rest of the program,
and robust indication of errors in spawn-style requires substantial growth in the spawn interface.
Additionally, modern operating systems such as Linux have features
which are difficult or impossible to fully exploit through fork-style or spawn-style.

Direct-style is rare in general, but relatively common on academic operating systems.
% insert citations to direct-style papers here
Direct-style involves creating a new child process in a non-running state,
and then calling individual syscalls from the parent to initialize the child.
Direct-style does not constrain the calling process,
easily branches at any step of the initialization,
and can create any kind of process supported by the underlying system.

Our contribution is an implementation of direct-style process creation for Linux,
making use of our work on the rsyscall project.
The rsyscall project develops language-specific ``process-independent'' syscall libraries for Linux,
which replace libc syscall wrappers and allow a program to perform cross-process syscalls.
rsyscall currently has best support for Python,
but can be ported to other languages;
the examples we show in this paper will be in Python,
but generalize easily to other languages.

In our implementation, a parent process calls the \texttt{clone} syscall
to create a child process which shares memory and other resources with the parent,
and which runs no user code.
The parent then uses \texttt{rsyscall}
to call arbitrary syscalls in the child process to initialize it;
for example, the parent might call \texttt{chdir} to change the child's current working directory.
After the process is fully initialized,
the parent can call \texttt{execve} in the child,
causing the process to start running an arbitrary executable
and detaching the child from the control of \texttt{rsyscall}.

A direct-style \texttt{clone} has all the advantages of direct-style process creation
over Linux's fork-style \texttt{fork} syscall and spawn-style \texttt{posix\_spawn} libc function.
Direct-style \texttt{clone} can be used correctly and performantly from any process.
Cross-process syscalls made with rsyscall individually report errors, as normal, and can be easily branched on.
Arbitrary processes involving complex namespaces and initialization can be created
using techniques that are impractical with \texttt{fork}.
The performance cost from high-level code varies, but the underlying primitives are substantially faster.

We have used direct-style process creation on Linux to do TODO and TODO,
and we've benchmarked our implementation as TODO.
Direct-style \texttt{clone} has minimal performance cost in normal situations,
and can outperform \texttt{fork} in some cases.
We have deployed direct-style process creation in production internally at Two Sigma.

Our implementation is distributed as a feature of \texttt{rsyscall},
and is immediately deployable on recent Linux kernels.
rsyscall is open source and available from \url{https://github.com/catern/rsyscall}.

This paper is organized in several sections.
In section \ref{background}, we provide background on fork-style, spawn-style, and direct-style process creation,
and consider their advantages and disadvantages in more depth.
In section \ref{examples} we show examples of usage of direct-style process creation in Python on Linux,
including use of process features which are difficult to use on Linux with \texttt{fork} or \texttt{posix\_spawn},
and briefly examine the user API.
In section \ref{implementation}, we explore our implementation in depth,
and discuss several difficult process-creation related details of \texttt{rsyscall}'s implementation.
In section \ref{evaluation}, we evaluate, somehow (TODO).
In section \ref{future_work},
we list several directions for future work related to direct-style process creation and to \texttt{rsyscall}.
In section \ref{conclusions}, we briefly and optimistically conclude.

\section{Background}\label{background}
Processes are a widespread feature of operating systems,
with substantial variation in characteristics between systems.
One of the areas of variation is in process creation mechanisms.

On most systems today,
the interfaces for process creation
can be divided into "fork-style" and "spawn-style".
A few systems use a third style, which we refer to as "direct-style".
\subsection{Fork-style}
Fork-style interfaces are those that follow the model used by Unix's fork \cite{forkhist}.
A process calls fork, and a copy of that process is made,
with the return value of fork being the only differentiation between the process and its copy.
The same program continues executing in both processes.
Typically, the program will case on the return value to determine if it's running in the parent or the child,
and run process setup and initialization code if the program is running in the child,
calling various syscalls to set up the state of the child.

Fork has a number of flaws,
and much ink has been spilled detailing them.
We'll cover the most prominent briefly here,
but see Baumann 2019 \cite{forkroad} for a more detailed evaluation.

Fork, when called successfully, returns twice,
creating two different processes which execute concurrently.
This immediately makes fork difficult to integrate;
for example, errors can't be easily communicated between the two processes.
If there is an error in the initialization of the child process,
the part of the program running in the child process
needs to use some form of inter-process communication (IPC) to communicate it back to the parent process.
The child process can run arbitrary code to make decisions about its initialization,
but that code needs to use IPC if it wishes to communicate with the rest of the program,
which is still in the parent process.

The most common complaint about fork is its poor performance \cite{forkroad}.
It copies many attributes about the parent process when creating the child process,
including setting up copy-on-write memory-mappings in the child process.
This becomes slower as the parent process has more memory-mappings,
eventually taking a significant amount of time.
This copying is required to robustly implement fork's model,
where the same program continues executing without changes in both the parent process and child process;
the copy of the program running in the child process might access any memory at any time.

Multi-threaded programs generally cannot use fork safely.
Typical Unix fork implementations duplicate only the thread calling fork from the parent process to the child process.
In a multi-threaded program, this can cause various issues;
for example, another thread might be holding a memory allocator lock at the time of the fork,
which in the child process will never be unlocked,
causing the child process to deadlock if it tries to allocate memory.
Some thread libraries provide partial mitigations for this issue,
but it's up to user code to make use of those mitigations.
In combination with fork's poor performance in large-memory programs,
a user of fork must think carefully
about the characteristics of the process from which fork is being called.
\subsection{Spawn-style}
Spawn-style interfaces are those that follow the model used by \texttt{posix\_spawn} \cite{posix_spawn}
and Windows' \texttt{CreateProcess} \cite{create_process}.
All the details about the new process are provided up-front as arguments to a syscall,
which creates the new process from a mostly-clean slate, initialized with the provided details.

Spawn-style interfaces typically still transparently copy some details from the parent process;
for example, various security contexts,
and any other process attribute which is not explicitly specified in the arguments to the spawn interface.
The key difference between fork-style and spawn-style is not how much they copy;
it is how the attributes which are not copied are specified:
by mutation from code running in the new process in the case of fork-style,
or by explicit argument passing in the case of spawn-style.

Spawn-style interfaces lift the constraints on the calling process that fork-style interfaces impose.
Since spawn-style interfaces don't run user code in the new process during initialization,
there is no need to copy memory,
and there are no inherent issues when the interface is used from a multi-threaded program
(though operating systems which implement spawn-style interfaces on top of \texttt{fork} can still have bugs).

However, the arguments that can be provided to a spawn-style process creation syscall
do not cover all the possible attributes that one might want to set for the new process.
Most systems have a large number of syscalls which can mutate the state of a process during its lifetime;
for a spawn-style interface to work in all scenarios,
all those possible mutations must be reproduced in the interface.
For example, the \texttt{posix\_spawn} function provided by glibc does not support creating processes in new namespaces,
as is required for container functionality on Linux.

Spawn-style process creation also does not allow for conditional logic during the setup;
if the setup of the new process encounters an error at some point,
the only option is to return from the entire spawn call with an error.
Such errors returned from spawn-style calls
are typically much less informative
than the errors returned by the syscalls which directly mutate the process attributes.
Other forms of conditional logic are also impossible in spawn-style;
one modification to the process cannot depend on the result of some other modification.
\subsection{Direct-style}
A few academic operating systems, such as KeyKOS \cite{keykos}, seL4 \cite{sel4}
and some others \cite{exokernel} \cite{fuschia} \cite{singularity},
use another style of process creation, which we refer to as "direct-style".

In direct-style, a process is created by its parent in a non-running state,
and then supplied with various resources,
and then started running once it is fully set up.
In operating systems with true direct-style process creation,
the syscalls that can mutate a process
take explicit arguments to indicate which process they should operate on.
In this way, the same syscalls that can mutate a process while it is running,
are called by the parent process to mutate the process while it is being set up.

We refer to this as "direct-style" process creation,
because the parent creating the process operates on it directly and imperatively
rather than dispatching a distinct unit of code to perform setup from inside the context of the new process,
as in fork-style,
or building up a declarative specification of what the new process should look like ahead of time,
as in spawn-style.

Since everything happens directly from the parent process,
process initialization is compatible with a variety of techniques
which were otherwise incompatible with process creation.
For example, a program could provide a UI which directly customizes the new process,
or might initialize several processes at once and share data between them.

Most significantly,
since all operations are performed through syscalls called directly from the main program,
errors are indicated in the same way as any other syscall error.
This is unlike fork-style, which needs IPC to indicate errors,
and unlike spawn-style, which typically indicates errors at a very coarse-grained level.

Like fork-style interfaces,
direct-style interfaces can set up arbitrary attributes in the new process.
Any attribute that can be changed by a syscall
can be manipulated by a direct-style interface,
just as with fork-style interfaces.
This is unlike spawn-style interaces,
which only can change attributes that are supported by the interface.

Like spawn-style interfaces,
direct-style interfaces have no constraints on the calling process.
No user code runs in the child process,
so, unlike fork-style interfaces,
direct-style interfaces can be used from multi-threaded or large-memory processes without issues.

Direct-style can be more complex to use;
it is most typically used in capability-oriented operating systems,
where a great deal of resources and information must be provided explicitly to initialize the new process.
In a truly capability-oriented operating system,
nothing is copied implicitly to the new process;
everything must be explicitly specified.
This can appear more complex
when compared to process creation on Unix systems or Windows.

However, we believe that a natural port of direct-style process creation to Linux
provides an interface that is just as simple to use as fork-style or spawn-style on Linux,
without their disadvantages.
Such an interface, like fork-style and spawn-style,
would implicitly copy some attributes to the new process,
rather than being capability-oriented and requiring all attributes to be specified explicitly.
This is the kind of direct-style interface that we contribute in this paper.

\section{Background on the use of processes}
Why is it important to have a high-quality interface for creating processes?
Processes are already widely used;
most software is distributed as an executable which runs in a dedicated process.
This basic usage of processes can be performed with even a complex and inefficient process creation interface.
But processes have many uses beyond this simple and widespread one;
here we examine some more sophisticated applications of processes,
which benefit from a better process creation interface.
\subsection{Abstraction over resources}
In Unix, the mechanism of file descriptor inheritance
allows a process to be provided a resource by its creator,
while abstracting over the precise identity of that resource.[fn:ucspi]
For example, a process can be provided a file descriptor,
which can correspond to any file in a filesystem,
without the process being aware of what specific file it is accessing.
This is further enhanced by Unix's "everything is a file" design;
the passed file descriptor could also be a pipe, a network connection, or some other resource,
without the process knowing.
As another example,
a process can be provided a socket file descriptor on which it can call \texttt{accept} to receive connections,
without being aware of whether those connections come from the internet or from a local Unix socket.[fn:ucspi]
This abstraction mechanism is the basic principle of pipelines and redirection in the Unix shell,
but it is rarely used outside of the shell.
\subsection{Customization without explicit support}
In many systems,
it's possible to modify a process's view of nominally "global" resources.
In Unix-derived systems, this ability is most influentially provided in Plan 9[fn:plan9],
which allows each process to customize its view of the filesystem with private mounts and union directories[fn:plan9ns].
In Linux, these concepts were implemented as per-process namespaces[fn:linuxns].
Fundamentally,
this allows customizing a process's environment and therefore a program's behavior,
without having to write explicit support code for customization.
For example, Plan 9, unlike most other Unix-derived systems,
did not have a \texttt{PATH} environment variable which was searched by code in the process to find executables;
instead, each process was executed with a \texttt{/bin} directory at the root of the filesystem,
which was a union of many other directories,
and simply executed \texttt{/bin/foo} to run the program named \texttt{foo}.
In this way the set of executables provided to a process could be customized,
without any code to parse and handle \texttt{PATH} or any other executable-lookup-specific customization code.
\subsection{Sandboxing}
The basic isolation powers of processes are used to simplify application development:
it is beneficial to have a private virtual memory space when developing a stand-alone program.
But most systems have additional mechanisms of isolation between processes,
such as different privilege levels and access to global resources,
which can be used to provide a form of sandboxing.
For example, components which may exposed to hostile network requests
can be run in a separate process, at a lower privilege level than the main program;
in this way, even if an attacker gains control over that component,
the attacker will only have access to the lower level of privileges of that component,
rather than the full privileges of the main program.
\subsection{Capability-based security}
As a further development of process-based sandboxing,
the privileges of a process can be explicitly enumerated
in a capability-based security model.[fn:capsicum]
By using previously-mentioned resource passing mechanisms,
such as file descriptor inheritance or namespace manipulation,
and by disabling the process's access to global resources such as the shared filesystem,
we can enforce that all resources used by the process are passed at creation time.
\subsection{Non-shared-memory concurrency}
Processes run concurrently,
which allows exploiting parallelism in the hardware.
Since processes don't share memory,
they can provide a less complex parallel programming environment
than shared-memory thread-based approaches.
The most popular parallel programming environment in existence today is the Unix shell,
which obtains its parallelism by running multiple processes connected via pipes.
The Unix shell has a relatively constrained form of parallel processing,
but it's also possible to create more complex webs of parallel processes,
where, for example, one process might take multiple inputs over multiple pipes,
or produce multiple outputs.
\subsection{Conclusion}
These techniques, and more, are available through the process interface.
Most software would benefit from abstraction over resources, sandboxing, and parallelism.
Yet these features of processes are used only rarely.
There are multiple reasons for this,
but one of the primary reasons is the complexity of current process creation interfaces.

Many of these techniques are used today by specialized software and services.
Often, such software only allows use of one of these techniques;
for example, the Unix shell allows piping together process, but not namespacing them;
container systems allow sandboxing processes, but not piping them together.
By delegating these features to specific separate services,
we lose the ability to use them in combination.

By improving the process creation interface,
we can make it possible both for programs to directly manipulate processes to use these techniques,
and to use and share composable libraries which use these techniques.
We believe this potential justifies the investment of substantial effort
into improving the process creation interface.
\section{Overview}\label{examples}
In most operating system APIs, including the ones available on Linux,
when a program makes a syscall, it implicitly operates on the current process.
To perform direct-style process creation on Linux,
we need a Linux API where we instead explicitly specify in which process we want to make a syscall.

\texttt{rsyscall} provides this API for Linux.

The \texttt{rsyscall} project develops
language-specific, object-capability-model, low-abstraction libraries for the Linux syscall interface,
bypassing libc.
Unlike POSIX C libraries such as glibc or musl,
an rsyscall library is organized based on the object-capability model.
The capabilities to make syscalls in any given process are reified as language objects.
If the capability to make syscalls in a specific process is not passed (in some way) to a function,
then that function cannot make syscalls in that process.

Due to this design, an \texttt{rsyscall} program can make use of multiple processes at once,
by manipulating capabilities for multiple processes.
The relevant two capabilities for this paper are the initial capability to the ``local'' process,
and capabilities to child processes.
The ``local'' process is the one which hosts the runtime for the running program,
and in which a legacy libc would implicitly make syscalls;
every rsyscall program starts with the ``local'' process
and uses it to bootstrap the initial capabilities for other processes.

From the perspective of an program using rsyscall,
there are multiple processes which are under its control,
through multiple capabilities.
We use the term ``thread'' to refer to all such controlled processes,
including the main "thread" on which the program is running.
On Linux, the shared-memory "threads" provided by libraries such as \texttt{pthreads}
are implemented as processes,
and their lifetime and execution is completely controlled by a single program.
The same is mostly true of the controlled-process "threads" provided by rsyscall.
The most significant difference is that rsyscall "threads"
do not run their own code concurrently with the main program;
nevertheless rsyscall "threads" do provide the opportunity for parallel execution of system calls,
and so the terminology provides useful intuition.

To perform direct-style process creation,
we use \texttt{rsyscall} to call \texttt{clone} on the ``local'' thread to create a new child thread.
We call various syscalls in the child thread to mutate it until it reaches the desired state,
at which point we can call \texttt{execve} in the child thread to start it running on its own.
After that, the process is no longer under our control, and it functions like any other child process.
(Hence we call it ``process'' rather than ``thread'')
The child process can be monitored using normal Linux child monitoring syscalls from our main thread,
such as \texttt{waitid},
which we can also do with \texttt{rsyscall}.

\begin{lstlisting}[float,language=Python,label={basic},caption={Creating a new process, changing CWD, and execing}]
child = local_thread.clone()
child.chdir("/etc")
child.execve(ls_path, [ls_path], local_thread.environ)
\end{lstlisting}

Listing \ref{basic} shows this in action in Python.
We create a new child thread using \texttt{clone} on \texttt{local_thread},
and receive back a capability to control it.
We pass no flags to use \texttt{clone}'s defaults
(which essentially match \texttt{fork}) to create the process.
We call \texttt{chdir} in the thread to change its working directory to \texttt{/etc}.
Then we call \texttt{execve},
passing the path of the \texttt{ls} executable,
the arguments to the new executable,
and the unchanged set of environment variables.
The path of the executable is passed as the first argument,
as is traditional in Unix.
Calling \texttt{execve} consumes the capability,
and later use of this child thread capability will fail with an exception.

The API for direct-style process creation on Linux is not new;
it is the same Linux system call API that systems programmers are already familiar with.
The novelty is constrained to the newly explicit process capabilities;
once that is understood, the rest of the API is mundane,
and the representation of these programs is obvious in any language.

The familiar and mundane nature of the API is important not just for the learning curve,
but also for expressivity.
Since the API is simply normal Linux system calls,
anything that could be expressed with fork and syscalls
can also be expressed with direct-style clone.
\section{Applications}
In this section,
we'll discuss several more advanced uses of processes,
using more sophisticated Linux features,
and accompany them with Python examples using direct-style process creation.
As always, since these examples are using the Linux system call API directly,
they can be translated straightforwardly into any language with an \texttt{rsyscall} API.

\subsection{Example: Passing down FDs}
\begin{lstlisting}[float,language=Python,label={fds},caption={Passing down FDs}]
child = local.thread.clone()
sock = child.socket(AF.INET, SOCK.DGRAM)
# Bind the socket to a sockaddr_in;
# the sockaddr is written to memory accessible by the child with child.ptr
sock.bind(child.ptr(SockaddrIn(0, 0)))
sock.listen(10)
sock.disable_cloexec()
child.execv(executable_path, [executable_path, "--listening-socket", str(int(sock))])
\end{lstlisting}
In Listing \ref{fds}, we create a new thread,
then create a listening socket bound to a random port in that thread,
then call exec, 
passing down the socket by disabling cloexec and passing its file descriptor number as an argument to the new program.

File descriptors, here, are object oriented and have relevant syscalls as methods.
They make syscalls in the process they are created in by default;
we can create more objects referring to the same file descriptor from different processes
if we want to make the syscalls from another process.
%% **** note to self: concepts introduced in this example
%% disable_cloexec
%% socket creation
%% using int on sock
%% .ptr
\subsection{Example: Pipeline}
\begin{lstlisting}[float,language=Python,label={pipe},caption={Creating a pipeline}]
# create the pipe
pipe = local.thread.pipe()
child1 = local.thread.clone()
# inherit the write-end of the pipe to child1, and replace child1.stdout with it
child1_pipe_write = child1.inherit_fd(pipe.write)
child1_pipe_write.dup2(child1.stdout)
child1_proc = child1.execv(yes_path, [yes_path])
child2 = local.thread.clone()
# inherit the read-end of the pipe to child2, and replace child2.stdin with it
child2_pipe_read = child2.inherit_fd(pipe.read)
child2_pipe_read.dup2(child2.stdin)
child2_proc = child2.execv(head_path, [head_path, "-n", "15"])
\end{lstlisting}
In Listing \ref{pipe},
we do the same as the Unix shell pipeline "yes | head -n 15".
We create a pipe,
then create two threads,
connect them with a pipe,
and exec a different program in each thread.

After a process is created with clone,
it may have inherited file descriptors;
here we inherit the pipe.
We make this inheritance explicit with \texttt{inherit\_fd},
a helper method on our thread object,
which takes a file descriptor from a different thread
and performs a runtime check that the file descriptor actually was inherited.
If so, it returns a new handle to the file descriptor which performs syscalls from the new thread.

Then we simply =dup2= as normal to replace child1's stdout with the write end of the pipe;
=dup2= disables CLOEXEC by default on the target.
%% **** note to self: concepts introduced in this example
%% dup2
%% pipe
\subsection{Example: Mount namespace}
\begin{lstlisting}[float,language=Python,label={mount},caption={Mount namespace}]
child = local.thread.clone(CLONE.NEWUSER|CLONE.NEWNS)
child.mkdir(rootdir/"proc")
child.mount(Path("/proc"), rootdir/"proc", "", MS.BIND, "")
child.chroot(rootdir)
child.execv(executable_path, [executable_path])
\end{lstlisting}
In Listing \ref{mount},
we make a new mount namespace and rearrange the filesystem tree for the child process.
We bind-mount /proc at /proc inside the chroot directory,
chroot into the directory,
and exec an executable which will run inside the chroot.
%% **** note to self: concepts introduced in this example
%% namespaces
%% path slash syntax sugar
\subsection{Example: Nested clone and network namespace}
\begin{lstlisting}[float,language=Python,label={nested},caption={Nested clone and network namespace}]
ns_thread = local.thread.clone(CLONE.NEWNET|CLONE.NEWUSER)

listening_child = ns_thread.clone()
sock = listening_child.socket(AF.INET, SOCK.DGRAM)
sockaddr = SockaddrIn(22, "127.0.0.1")
sock.bind(listening_child.ptr(sockaddr))
sock.listen(10)
sock.disable_cloexec()
child.execv(server_path, [server_path, "--listening-socket", str(int(sock))])

connecting_child = ns_thread.clone()
child.execv(client_path, [
  client_path, "--connect-address", str(sockaddr.address) + ":" + str(sockaddr.port)])
\end{lstlisting}
In Listing \ref{nested},
we make a process (\texttt{ns\_thread}) in a new network namespace.
Then, we create two more child processes of \texttt{ns\_thread},
which are also in the new network namespace.
This nested creation of child processes is fully supported,
like all other syscalls,
and allows us to set up complex graphs of processes and namespaces.

We bind to a privileged port on localhost inside the namespace,
and create one child to listen on that socket,
and another child to connect to it.
\subsection{Example: Starting a complex system}
\begin{lstlisting}[float,language=Python,label={miredo},caption={Starting a complex system}]
### create socket outside network namespace that Miredo will use for internet access
inet_sock = local.thread.socket(AF.INET, SOCK.DGRAM)
inet_sock.bind(local.thread.ptr(SockaddrIn(0, 0)))
# set some miscellaneous additional sockopts that Miredo wants
set_miredo_sockopts(local.thread, inet_sock)
### create main network namespace thread
ns_thread = local.thread.clone(CLONE.NEWNET|CLONE.NEWUSER)
### create in-network-namespace raw INET6 socket which Miredo will use to relay pings
icmp6_fd = ns_thread.socket(AF.INET6, SOCK.RAW, IPPROTO.ICMPV6)
### create in-network-namespace socket,
### which Miredo will use for unassociated Ifreq ioctls
reqsock = ns_thread.socket(AF.INET, SOCK.STREAM)
### create and set up the TUN interface
tun_fd, tun_index = make_tun(ns_thread, "miredo", reqsock)
### create socketpair which Miredo will use to communicate
### between privileged process and Teredo client
privproc_pair = ns_thread.socketpair(AF.UNIX, SOCK.STREAM)
### start up privileged process which manipulates the network setup in the namespace
privproc_thread = ns_thread.clone()
# We preserve NET_ADMIN capability over exec so that privproc can
# manipulate the TUN interface. add_to_ambient_caps is a helper function
# used because manipulating Linux ambient capabilities is fairly verbose
add_to_ambient_caps(privproc_thread, {CAP.NET_ADMIN})
# privproc expects to communicate with the main client over stdin and stdout
privproc_side = privproc_thread.inherit_fd(privproc_pair.first)
privproc_side.dup2(privproc_thread.stdin)
privproc_side.dup2(privproc_thread.stdout)
privproc_child = privproc_thread.execv(miredo_privproc_executable_path, [
    miredo_privproc_executable_path, str(tun_index)
])
### start up Miredo client process which communicates
### over the internet to implement the tunnel
# the client process doesn't need to be in the same network namespace,
# since it is passed all the resources it needs as fds at startup.
client_thread = ns_thread.clone(CLONE.NEWUSER|CLONE.NEWNET|CLONE.NEWNS|CLONE.NEWPID)
# lightly sandbox by unmounting everything except for the executable
# and its deps (known via package manager)
unmount_everything_except(client_thread, miredo_exec.run_client.executable_path)
# a helper function for preparing the fds that are passed as command line arguments
async def pass_fd(fd: FileDescriptor) -> str:
    client_thread.inherit_fd(fd).disable_cloexec()
    return str(int(fd))
client_child = client_thread.execv(miredo_client_executable_path, [
    miredo_client_executable_path,
    pass_fd(inet_sock), pass_fd(tun_fd), pass_fd(reqsock),
    pass_fd(icmp6_fd), pass_fd(privproc_pair.second),
    "teredo.remlab.net", "teredo.remlab.net"
])
\end{lstlisting}

In Listing \ref{miredo},
we show non-trivial code for launching a real application:
the Miredo IPv6 tunneling software.
We use a few helper functions in this listing to keep the attention focused on the interesting parts.

Miredo is separated into two components, a privileged process which sets up network interfaces,
and an unprivileged process which talks to the network.
With minimal modifications to Miredo,
we launch Miredo entirely unprivileged inside a user namespace and network namespace,
with all resources created outside and explicitly passed in.
\section{Implementation}\label{implementation}
\subsection{Basics about rsyscall}
Our main need for implementing direct-style process creation
is a robust system for cross-process syscalls.
The rsyscall project provides this.

The rsyscall project develops language-specific libraries
for the Linux syscall interface, with minimal abstraction.
The rsyscall project develops a collection of language-specific libraries tightly coupled to Linux,
which provide access to Linux syscalls with minimal abstraction,
bypassing libc.

Unlike POSIX C libraries, an rsyscall library is organized in a capability-oriented style;
the capability to make syscalls in a process is explicicitly passed around as a language object.
This organization allows an rsyscall program to be process-independent,
and straightforwardly make system calls in any of the processes under its control,
so-called ``cross-process'' system calls.

for 
These 
for the is an alternative, low-abstraction to the Linux
The rsyscall project is a toolkit for cross-process syscalls on Linux,
with several language-specific library implementations.

In this section, we'll give a brief overview of rsyscall,
and focus on implementation issues specific to process creation.

rsyscall can be conceptually divided in two parts:
the basic cross-process syscall primitive,
and a language-specific library built on top
to handle the complexities of manipulating resources across multiple processes.
The Python language-specific library has already been demonstrated above.
Such libraries only need to be able to call syscalls and explicitly specify a process in some way;
they are, for the most part, agnostic to how the cross-process syscall is implemented.

Using the Python library as an example,
it provides Python wrappers for Linux system calls and structs
which are type-safe using Python 3 type annotations and runtime checks
while still providing low-abstraction access to a large subset of native Linux functionality.
It also provides garbage collection for remote file descriptors, memory and other resources.
Such features are independent of the precise implementation of the cross-process syscall primitive.

On Linux \verb|x86_64|, a syscall is specified by a syscall number plus six register-sized arguments;
a syscall returns one register-sized value.
rsyscall's default implementation of cross-process syscalls sends those seven integers over a pipe,
and waits for a response on another pipe.
Processes are created running an infinite loop which, at each iteration,
reads a syscall request off the pipe,
performs that syscall,
and writes the return value back over the return pipe.
In this way, a cross-process syscall works much like a very primitive remote procedure call.

Many syscalls either take or return pointers to memory,
and require the caller to read or write that memory to provide arguments or receive results.
Therefore, an rsyscall library needs a way to access memory in the target process.
We implement this through another set of pipes,
by explicitly copying memory into and out of those pipes using the \texttt{read} and \texttt{write} system calls.
When we wish to read \texttt{N} bytes of memory at address \texttt{A} in the target process,
we first perform a \verb|write(memory_pipe, A, N)| in the target process,
and then read that data off the other end of the pipe in the parent process.
When we wish to write \texttt{N} bytes of data at address \texttt{A} in the target process,
we first write that data to the pipe in the parent process,
then perform a \verb|read(memory_pipe, A, N)| in the target process to copy that data from the pipe into memory.

ptrace provides an alternative means to perform arbitrary actions on other processes.
However, among other issues, it has the unavoidable substantial disadvantage of not permitting multiple ptracers.
A ptrace-based implementation would prevent using strace or gdb on rsyscall-controlled processes,
which is an unacceptable limitation for a general-purpose utility.

The \verb|process_vm_readv| and \verb|process_vm_writev| system calls
allow the caller to read and write memory from the virtual address space of other processes.
However, they require that the caller have specific credentials relative to the process being accessed,
which may not always be the case.
Additionally, these system calls are disabled if ptrace is disabled system-wide,
which is a niche but possible system configuration.
To ensure that rsyscall can be used for arbitrary purposes and on arbitrary systems, we avoided these calls.
\subsection{\texttt{clone}}\label{clone}
Now that we've established the basic operations which rsyscall provides,
let's consider the specific issues related to process creation and initialization.

There are three Linux system calls which create processes:
\texttt{fork}, \texttt{vfork} and \texttt{clone}.
\texttt{clone} provides a superset of the functionality of the other two,
so we focused our attention on \texttt{clone}.

\texttt{clone} (along with \texttt{fork}) creates a new process
which immediately starts executing at the next instruction after the syscall instruction,
in parallel with the parent process,
with its registers in generally the same state as the parent process.
\footnote{Note that =glibc= defines a wrapper for the clone syscall;
we are talking about the raw kernel syscall.}
In the style of Plan 9's \texttt{fork} syscall\cite{rfork}, which inspired \texttt{clone},
\texttt{clone} takes a mask of flags which determines whether several attributes of the new process
are either shared with, or copied from, the parent process.

\texttt{clone} only lets us change the stack register for the new process.
We would like to be able to set arbitrary registers for the new process,
so that we can control where it begins executing and the stack it executes on.
Fortunately, changing the stack is sufficient.

We ensure that the next instruction executed after any syscall
is (in x86 terms) a \texttt{RET};
this is always the case, so we have no need to special case the execution of \texttt{clone}.
Since we control the stack of the new process,
the \texttt{RET} will jump to a code address that we control.
We can then supply additional arguments to this code
by putting them on the stack.

We typically cause the new process to jump to a trampoline provided by the rsyscall library
which sets all registers to values found on the stack
and then jumps to another address.
\footnote{This is also a generally useful utility for hackers performing return-oriented-programming attacks;
but similar functionality exists in any standards-compliant C library,
so there is no increase in attack surface.}
With this trampoline,
we can provide a helper Python function that,
when given a function pointer following C calling conventions, and some arguments,
will prepare a stack for a call to clone such that the new process will call that function with those arguments.

With our new ability to call arbitrary C-compatible functions,
we can now call \texttt{clone} so that it launches a process running our infinite syscall loop,
which is implented in C and, as described in the previous section,
uses two pipes passed as arguments to receive syscall requests and respond with syscall results.

% TODO is this section necessary?
The addresses of these functions and trampolines are discovered through a linking procedure.
When the process being created is in the same address space as the main process which is running user code,
the location of the rsyscall library in memory, and the addresses of code within it,
are known through normal language-specific linking mechanisms.
However, when a process is created with a different address space,
such as when we establish a connection to a process after it's been started,
we need to perform linking to learn the addresses of functions.
This linking procedure is performed while bootstrapping the connection,
and involves the target process sending a table of important addresses to the connecting process.

After using \texttt{clone} to create a new process running our syscall loop,
most system calls can be called as normal.
The new process can be modified freely through chdir, dup2, and other system calls.
Out of system calls related to process creation,
only \texttt{execve} and \texttt{unshare} need substantial further attention.
\subsection{\texttt{exec}}
Eventually, most programs will want to call \texttt{execve} in the processes they create.
\texttt{execve} is unusual and requires careful design,
because when it is successful, it does not return.
Therefore we need a way to determine if \texttt{execve} is successful;
naively waiting for a response to the syscall request will leave us waiting forever.

One traditional means to detect a successful \texttt{execve} is to create a pipe before forking,
ensure both ends are marked \verb|O_CLOEXEC|,
perform the fork,
call \texttt{execve} in the child,
close the write end of the pipe,
and wait for EOF on the read end.
If the child process has neither successfully called \texttt{execve}, nor exited for some other reason,
then the write end of the pipe will still be open in the child process's fd table,
and the read end of the pipe will not return EOF.
But once the child process calls \texttt{execve} successfully,
\verb|O_CLOEXEC| will cause the write end of the pipe to be closed,
and the read end of the pipe will return EOF.

This trick works well with \texttt{fork};
but it's not general enough to work with \texttt{clone}.
Child processes can be created with the \verb|CLONE_FILES| flag passed to \texttt{clone},
which causes the parent process and child process to share a single fd table.
This means that when the parent process closes the write end of the pipe,
it will also be closed in the child process,
and the read end of the pipe will immediately return EOF,
regardless of whether the child has called \texttt{execve} or exited.

Fortunately, there is an alternative solution, which does work with \verb|CLONE_FILES|.
The \texttt{ctid} argument to \texttt{clone} specifies a memory address which,
when the \verb|CLONE_CHILD_CLEARTID| flag is set,
the kernel sets to zero when the child exits or execs,
and then, crucially, performs a futex wakeup on.
More specifically,
the kernel clears and does a futex wakeup on \texttt{ctid} when the child process leaves its current address space;
this precisely coincides with exiting or execing,
since those are the only way to change address space in Linux as of this writing.

A futex is a Linux-specific feature,
which is generally used for the implementation of userspace shared-memory synchronization constructs,
such as mutexes and condition variables.
The relevant detail for us here is that we can wait on an address
until a futex wakeup is performed on that address;
that means we can wait on \texttt{ctid} until the futex wakeup is performed,
and in this way get notified of the child process calling \texttt{execve}.

Unfortunately, futexes in current Linux integrate poorly:
There is no way for a single process to wait for more than one futex at a time,
and no way to monitor a futex with file-descriptor-monitoring syscalls such as \texttt{poll}.
The best we can do is create a dedicated child process for each futex we want to wait on,
and have this child process exit when the futex has a wakeup.
Monitoring child processes can be straightforwardly integrated into an event loop.

While slightly complex to implement, this solution works well.
We provide \texttt{ctid} whenever we call \texttt{clone},
and set up a process to wait on that futex.
Then, when we call \texttt{execve},
we wait for either the \texttt{execve} to return an error or the futex process to exit,
whichever comes first.
If the futex process exits,
and the child process doesn't itself exit,
we know that the child has successfully called \texttt{execve}.

If the futex process and child process both exit,
it's ambiguous whether the child process successfully called \texttt{execve};
this ambiguity is unfortunate, but it is also present in the pipe-based approach.
This is, we believe, the best solution currently available.

We would prefer for Linux to natively provide functionality to wait for a child's \texttt{execve}.
Some other Unix-like systems provide this;
kqueue, on FreeBSD, allows waiting for exec in arbitrary processes through kqueue's \verb|EVFILT_PROC|.
One approach for Linux would be to add a new \texttt{clone} flag to opt-in to receiving \texttt{WEXECED} events through \texttt{waitid};
note that a \texttt{waitid} flag alone is not sufficient,
since it's necessary to receive \texttt{SIGCHLD} signals for the \texttt{WEXECED} event if waiting for it from an event loop.
Alternatively, some way to wait for futex wakeups through a file descriptor could be added,
so we can use file-descriptor-monitoring syscalls to wait for the \texttt{ctid} futex;
such a feature used to exist in the form of \verb|FUTEX_FD|,
but was removed from Linux long ago due to race conditions in its design.
\subsection{Managing file descriptor tables}\label{fdtables}
As mentioned in the previous section,
the \verb|CLONE_FILES| flag can be passed to \texttt{clone}.
When this flag is passed,
the file descriptor table is shared between the parent process and child process.
The same file descriptors are open in both processes at the same numbers,
and if new file descriptors are opened in either process,
they are also visible in the other process.
This is simple to model,
and convenient for many purposes;
for example, the child process might be in a different network namespace from the parent,
and the shared file descriptor table would allow the child to bind a socket
and the parent to use it;
the file descriptor table is unshared once the child calls \texttt{execve}.

If \verb|CLONE_FILES| is not passed to \texttt{clone},
then \texttt{clone} has the same behavior as \texttt{fork}:
The new process has a new file descriptor table,
containing copies of all the file descriptors existing in the parent at the time of the system call.
This same behavior can also be triggered after process creation by calling \verb|unshare(CLONE_FILES)| or \texttt{execve};
if \verb|unshare(CLONE_FILES)| or \texttt{execve} (ignoring \texttt{CLOEXEC}, which we'll discuss later),
are called in a process currently sharing its file descriptor table with another process,
then after the call that process will have a new, private file descriptor table,
again with a copy of all the file descriptors existing at the time of the system call.

After a system call has copied all the file descriptors in the old table into the new table,
we need to decide which file descriptors we want to keep open in the new table,
and which file descriptors should be closed.
Keeping some file descriptors from the old table in the new table
is referred to as "file descriptor inheritance".
\subsubsection{Inheriting file descriptors}
The rule about which file descriptors stay open is simple:
We want to keep a file descriptor open in the new table
if there is a process using that file descriptor.
We track which processes are using which file descriptors as part of our file descriptor garbage collection system.
File descriptors are used through garbage-collected handles,
each of which is associated with a process.
If there is an existing handle for a (process, file descriptor number) combination,
this means that the file descriptor with that number in that process's file descriptor table is in use by that process.

This makes inheritance simple in the case of \verb|unshare(CLONE_FILES)|.
Every file descriptor in use by a process
is referenced by a (process, file descriptor number) combination;
these handles are still valid after a call to \verb|unshare(CLONE_FILES)|,
they simply now refer to the new file descriptor in the process's new file descriptor table.
So, these file descriptors will automatically be kept open;
that is, they will be automatically inherited into the new table.
This same mechanism can work across \texttt{execve},
if control over the process is re-established after the \texttt{execve},
and if we carefully manage \texttt{CLOEXEC}.

When a process is newly created,
no file descriptor handles yet exist which involve that process.
Once a handle is created with that process and some file descriptor number,
the referenced file descriptor will be inherited into (that is, kept open in) the new file descriptor table.
These handles can be created through the \verb|inherit_fd| function.

At the time of creating a new file descriptor table,
we perform some bookkeeping:
We make a list of all the file descriptors that existed in the old table
at the time of the creation of the new table.
These file descriptors are the ones which were copied into the new table.

The \verb|inherit_fd| function uses this list.
It takes as arguments a new process and a file descriptor handle from another process,
and checks (with the list) that that file descriptor was copied into the new process's file descriptor table
and hasn't since been closed.
If so, it creates a new handle for that (process, file descriptor number) combination.
This causes the file descriptor to stay open in the new file descriptor table,
and it can be used in the new process through the handle.
\subsubsection{Closing file descriptors after inheritance}
After file descriptor inheritance is complete,
we must promptly close other file descriptors we don't want to inherit.
Leaving these file descriptors open in the new table is a form of resource leakage.
It can also cause erroneous behavior.
For example, it's a common practice to close the write end of a pipe
and expect an EOF on the read end;
if the write end is copied into the new file descriptor table before being closed,
and the write end is never closed in the new table,
the read end will never get an EOF.

However, we can't simply close all other file descriptors.
The possibility of implicit inheritance of file descriptors is a traditional Unix feature,
which is useful in a wide variety of situations,
in much the same way as implicit inheritance of environment variables;
it can allow a resource to be passed down a process hierarchy without intervening programs being aware.

Here is where \texttt{CLOEXEC} becomes relevant:
\texttt{CLOEXEC}, in practice,
is a tag, set by userspace, for file descriptors which should not be implicitly inherited.
If \texttt{CLOEXEC} is set on a file descriptor,
we should close that file descriptor if we don't explicitly want to inherit it;
if \texttt{CLOEXEC} is not set, we should not close the file descriptor,
but instead we should allow it to be implicitly inherited.

This interpretation of \texttt{CLOEXEC} is a consequence of \texttt{CLOEXEC}'s primary purpose:
Managing inheritance of file descriptors over \texttt{execve},
in programs where the caller of \texttt{execve}
doesn't know all the file descriptors that may have been opened by the rest of the program;
this describes, in practice, all programs which call \texttt{execve}.
When \texttt{CLOEXEC} is set on a file descriptor,
it will not be copied into the new file descriptor table created after \texttt{execve}.
Thus \texttt{CLOEXEC} is a way of saying, before an \texttt{execve},
that this file descriptor will not be used by the new program after an \texttt{execve},
and therefore should not be inherited.
Since, in a general purpose program, \texttt{execve} may be called at any time,
libraries must have \texttt{CLOEXEC} set correctly at all times.
So we can check \texttt{CLOEXEC} at any time to see whether a file descriptor should be inherited or not,
even if the file descriptor is used by a library unrelated to rsyscall.

The implementation of closing non-inherited file descriptors is then simple:
We close all file descriptors which have \texttt{CLOEXEC} set
and which aren't referenced by an rsyscall file descriptor handle.
We also clear the list of copied file descriptors which \verb|inherit_fd| uses.
We do this closing operation in userspace;
a syscall to perform this would be a useful addition to Linux.

\texttt{CLOEXEC} is set by default on all file descriptors opened by rsyscall,
though it may be unset by a user program.
Many user programs will unset \texttt{CLOEXEC} on some file descriptors immediately before calling \texttt{execve}
so that the executable they run will inherit those file descriptors.
The new program run by the \texttt{execve} will typically immediately set \texttt{CLOEXEC} again.

An additional argument to the \texttt{execve} syscall
which allows specifying an explicit list of file descriptor numbers to inherit despite the \texttt{CLOEXEC} flag being set
would allow programs to avoid this pointless behavior of un-setting and then immediately re-setting \texttt{CLOEXEC}.
It would also, more significantly, allow programs to inherit file descriptors across \texttt{execve}
while the file descriptor table is shared;
unsetting \texttt{CLOEXEC} while the file descriptor table is shared will cause race conditions,
because the other processes sharing the file descriptor table might call \texttt{execve} at any time,
and so the file descriptor table must first be unshared before unsetting \texttt{CLOEXEC} and calling \texttt{execve}.
\section{Evaluation}\label{evaluation}
\subsection{performance ideas}
Can mmap a bunch of stuff in a process and then create some children,
and show that without having to copy page tables (due to \verb|CLONE_VM|), it's faster.

Can make the classic chart of fork vs glibc's \verb|posix_spawn| vs direct-style.
\subsection{abstraction ideas}
Implementing Python's popen/subprocess with rsyscall?
But that's not fair, theirs is portable, we'll be way simpler for free.

Rewrite some existing system with rsyscall?
Write a shell?
\section{Future work}\label{future_work}
\subsection{Other applications of rsyscall}
Most avenues of future work focus on rsyscall.
rsyscall was not developed solely for the purpose of this paper,
and it has many uses unrelated to direct-style process creation,
such as asynchronous system calls, exceptionless system calls\cite{flexsc}, cross-host operations, among others.
We are actively exploring such applications,
as well as broadening rsyscall's language support.
\subsection{Kernel support}
rsyscall's cross-process syscalls can be performed entirely in userspace,
which has substantial benefits for deployability.
Nevertheless, direct kernel support for creating a stub process and performing syscalls in the context of that process
may provide efficiency benefits, as well as reducing userspace-visible complexity.

Several other aspects of our implementation would be improved by kernel support.
We discussed these in the implementation section;
in brief, we would most benefit from kernel support for
detecting when a child process finishes =execve=,
closing all =CLOEXEC= file descriptors except for an explicitly specified list,
and explicitly specifying a list of =CLOEXEC= file descriptors to inherit over =excve=.
Implementing these features in the kernel in a generally useful way, and upstreaming them,
is an important direction for our future work.
\subsection{Portability to other Unix systems}
Other non-Linux systems
could adopt the techniques of this paper
to provide direct-style process creation.
Currently, our focus is on Linux,
but others may wish to explore porting these techniques to other operating systems.
\subsection{Large scale open source usage}
We have made use of the techniques described in this paper
in proprietary software at Two Sigma.
While this gives us personally greater confidence in these techniques,
it would be better to use them in a publicly available, open source system.
Either porting an existing system to use these techniques,
or using these techniques to create a substantial new system from scratch,
would provide a meaningful demonstration of the viability of these techniques.
% \subsection{file descriptor lifetime management}
% % TODO this is hard to explain, probably best to drop this
% Keeping file descriptors open if and only if there is a specific process using that file descriptor
% is not the only possibility.
% Keep in mind that "a process using the file descriptor" is a slight abuse of terminology;
% processes don't use file descriptors, programs do, and in our system there is only one program.
% To reflect this, we could instead more directly use file descriptors in a process-agnostic way;
% this would support the creation of objects
% which work transparently across multiple processes which share a file descriptor table.
% Such objects would automatically use the relevant process to perform the syscalls for any specific operation.
% A process-centric view of file descriptors instead forces each object to be associated with one process.
% Nevertheless, we found that the process-centric perspective better matches the existing intuitions of users,
% especially those with prior experience in programming with processes.
% We hope that future systems for multi-process programming
% might explore an object-centric approach for managing resources.
\section{Conclusions}\label{conclusions}
Direct-style process creation is much less known and much less used than fork-style and spawn-style.
We have implemented direct-style process creation for Linux.
Our implementation is immediately deployable on today's Linux systems.
We have discussed various applications of processes,
and demonstrated the use of Linux direct-style process creation
to implement them.
We hope that this work will help encourage more use of the process abstraction,
which, though widespread,
is still not used to its full potential.

\bibliographystyle{ACM-Reference-Format}
\bibliography{bibliography}

\end{document}
